\documentclass[a4paper,class=article,border=10pt,tikz]{standalone}

%mypackages
\usepackage{pythontex}
\usepackage{pgfplots}
\usepackage{amsmath}
\usepackage{titlesec}
\usepackage{tikz}
\usetikzlibrary{shapes.geometric}
\usetikzlibrary{positioning}
\usetikzlibrary{snakes,calc,positioning,patterns,angles,quotes,decorations.pathmorphing,decorations.markings}
% \titleformat{<command>}[<shape>]{<format>}{<label>}{<sep>}{<before-code>}[<after-code>]
%\titleformat{\section}{\normalfont\Large\bfseries}{\thesection.}{10pt}{}
% \titlespacing{<command>}{<left>}{<before-sep>}{<after-sep>}
%\titlespacing{\section}{0pt}{14pt}{7pt}

%\titleformat{\subsection}{\normalfont\itshape}{\thesubsection.}{10pt}{}
%\titlespacing{\subsection}{0pt}{12pt}{6pt}
% set font encoding for PDFLaTeX, XeLaTeX, or LuaTeX
\usepackage{ifxetex,ifluatex}
\if\ifxetex T\else\ifluatex T\else F\fi\fi T%
  \usepackage{fontspec}
\else
  \usepackage[T1]{fontenc}
  \usepackage[utf8]{inputenc}
  \usepackage{lmodern}
\fi

\usepackage{hyperref}


\title{Title of Document}
\author{Name of Author}

% Enable SageTeX to run SageMath code right inside this LaTeX file.
% http://doc.sagemath.org/html/en/tutorial/sagetex.html
% \usepackage{sagetex}

% Enable PythonTeX to run Python – https://ctan.org/pkg/pythontex
% \usepackage{pythontex}

\begin{document}


\newpage

\begin{pycode}
left_end=0
length=0
left_sup=0
right_sup=2
mass_point=-1
vibr_point=10

list=[left_end,left_sup,right_sup,mass_point,vibr_point,length]

max=max(list)
min=min(list)
height=max-min
\end{pycode}

\begin{pysub}
\begin{tikzpicture}
\tikzstyle{spring}=[thick,decorate,decoration={zigzag,pre length=0.3cm,post length=0.3cm,segment length=0.3cm}]
\tikzstyle{ground}=[fill,pattern=north east lines,draw=none,minimum width=0.75cm,minimum height=0.3cm]

\coordinate (origo) at (0,0);
\coordinate (left_end) at (!{min}cm,0cm);

\node (cantilever) [label={above:$M,J$},fill=gray,draw,outer sep=0pt,thick,minimum width=!{max-min}cm, minimum height=!{0.01*(max)}cm,xshift=!{0.5*(max-min)}cm] at (left_end) {};
\coordinate (left_sup) at (!{left_sup}cm,0cm);
\coordinate (right_sup) at (!{right_sup}cm,0cm);
\coordinate (vibr_point) at (!{vibr_point}cm,0cm);
% \node[thick,regular polygon,regular polygon sides=3, draw,anchor=north,yshift=!{-0.005*max}cm] at (left_sup);
\node[thick,regular polygon,regular polygon sides=3, draw,anchor=north,yshift=!{-0.005*max}cm] (polygon) at (right_sup) {};

% \draw[ultra thick,<-,anchor=south,yshift=!{0.005*max}cm] (!{mass_point}cm,0cm)--(!{mass_point}cm,1.5cm);
% \node[draw,circle,fill=red] (vibration_point) at (vibr_point);
\draw [spring,anchor=south]  (!{min}cm,!{-0.01*(max)}cm) ++ (0cm,!{0.005*height}cm) -- (!{min}cm,-1.5cm) node[left,midway] (left_spring) {$k_r$};
\draw [spring]  (!{max}cm,!{-0.01*(max)}cm) ++ (0cm,!{0.005*height}cm) -- (!{max}cm,-1.5cm) node[left,midway] (right_spring)  {$k_c$};
\node[draw,circle,fill=red,label={[yshift=0.4cm,xshift=-0cm]:$\varphi_r$}] (rocker_rot_point) at (right_sup) {};

\node (ground1) [ground,anchor=north west,xshift=-0.5cm,yshift=-0cm,minimum width=1cm] at (polygon.south) {};
\node (ground2) [ground,anchor=north,xshift=0.3cm,yshift=-0.75cm,minimum width=1cm] at (left_spring) {};
\draw (ground1.north west) --(ground1.north east);
\draw (ground2.north west) --(ground2.north east);

\node[draw,circle,minimum size=6cm,yshift=-1.5cm,anchor=north] (cam) at (vibr_point) {};
\node[draw,circle,fill=black,xshift=-0.5cm,yshift=0.75cm,label={[yshift=0.4cm,xshift=-0cm]:$\varphi_c$}] (cam_rot_point) at (cam) {};

\node[thick,regular polygon,regular polygon sides=3, draw,anchor=north] (cam_support) at (cam_rot_point) {};
\node (ground3) [ground,anchor=north,xshift=0cm,yshift=-0.2cm,minimum width=1cm] at (cam_support) {};
\draw[] (ground3.north west) --(ground3.north east);
\draw[thick, ->,] (cam_rot_point)[xshift=0.4cm,yshift=-0.2cm] arc (-35:215:0.5cm) ;
\draw[thick, <-,] (rocker_rot_point)[xshift=0.4cm,yshift=-0.2cm] arc (-35:215:0.5cm) ;
\draw (cantilever.west)--++(0cm,1.5cm) node (long_line_left) {};
\draw (cantilever.east)--++(0cm,1.5cm) node (long_line_right) {};
\draw[<->,transform canvas={yshift=-0.3cm}] (long_line_left.north)--(long_line_right.north) node[midway,above] {$L$};


\draw (ground2.south)--++(0cm,-1cm) node (short_line_left) {};
\node (short_line_right) at (ground1|-short_line_left.south) {};
\draw(ground1.south)--(short_line_right);
\draw[<->,transform canvas={yshift=0.3cm}] (short_line_left.south)--(short_line_right.center) node[midway,above] {$L_1$};
\end{tikzpicture}
\end{pysub}




\end{document}
