\documentclass[a4paper,14pt,twoside]{extreport}
%\usepackage{../../Preambly}
\usepackage{graphicx}
\usepackage{pythontex}
\usepackage{amsmath}
\usepackage[rawfloats=true]{floatrow} 
\begin{pycode}
from sympy import *

F, m, g, f, A, c, v, rho, P, E, t  = symbols('F, m , g, f, A, c, v, \rho, P, E, t')

\end{pycode}
\mathcode`\*=\number\cdot
%%%%%%%%%%%%%%%%%%%%%%%%%%%%%%%%%%%%%%%%%
\begin{document}
\begin{titlepage}
\centering

\includegraphics[width=\textwidth]{simrname.png} 

\vspace{8cm}
{\Large \bfseries Project on Electric and Hybrid Drives}

\vspace{2cm}
{\large \textit{Daniel Ruszczyk 297580}}


\vspace{6cm}
April 13, 2020
\end{titlepage}
%%%%%%%%%%%%%%%%%%%%%%%%%%%%%%%%%%%%%%%%%
\tableofcontents
\setcounter{chapter}{-1}
\chapter{Glossary}
\begin{description}%
\item[$m$]%
Vehile mass%
\item[$r_{d}$]%
dynamic radius%
\item[$A$]%
Front area of the vehicle%
\item[$c_{x}$]%
Shape coefficient%
\item[$f_{r}$]%
Friction coefficient%
\item[$g$]%
Gravity%
\item[$v$]%
Vehicle velocity%
\item[$i$]%
Current%
\item[$\eta$]%
Efficiency%
\item[$\phi$]%
Flux%
\item[$R$]%
Resistance%
\end{description}%
% -----------------------------------------
\chapter{Introduction}
The goal of that work is to examine the electric vehicle performance based on driving cycle. The scrutiny was performed using Matlab/Simulink Software.
\section{Driving cycle}
\begin{figure}[H]
  \centerline{\includegraphics[width=\textwidth]{driving_cycle.png}} 
  \caption{Driving cycle as time function}
\end{figure}
\section{Vehicle data}
Vehicle:
\begin{itemize}
    \item Vehicle weight $M = 1420 ~ kg$
    %\item Gross vehicle weight rating  $GVMR = 1680 kg$
    \item Dynamic radius $r_{d} =  0.313 ~ m$
    \item Frontal area $A =2.42  ~ m^2$
    \item Air drag coefficient $c_{x} = ~ 0.35$
    \item Rolling resistance $f_{r} = 0.01$
\end{itemize}
% -------------------------------- Simulink
\chapter{Simulink model}
\section{Resistance Force}
\begin{center}
\underline{Inertial resistance} \\
Inertial resistance is function of velocity change in time.\\
\begin{equation}
\large{ F_{inert} = m *\frac{dv}{dt} }
\end{equation}


\underline{Rolling resistance} \\
Rolling resistance is a function of speed. \\
\begin{equation}
\large{F_t = m*g*f_t}
\end{equation}
$f_t = f_0(1+k_vv^2)$ , where $k_v = 5 \cdot 10^{-5}$

\normalsize
\bigskip
\underline{Air drag resistance} \\
Aerodynamic drag defined as the function of speed. \\
\begin{equation}
\large{F_{drag} = \frac{1}{2}*\rho*A*c_x*v^2}
\end{equation}

\normalsize
\bigskip
\underline{Resistance Force} \\\
The resistance force is defined as the sum of rolling resistance and the aerodynamic drag.\\
\begin{equation}
\large{F_{op} = F_{inert} + F_t + F_{drag} }
\end{equation}

\normalsize
\bigskip
Finally the total resistance power is equal
\begin{equation}
\large{P_{wheel} = F_{op}*v }
\end{equation}
\end{center}

\begin{figure}[H]
  \centerline{\includegraphics[width=\textwidth]{partial_resistances.png}} 
  \caption{Partial resistance forces as time function}
\end{figure}


\begin{figure}[H]
  \centerline{\includegraphics[width=\textwidth]{total_resistance.png}} 
  \caption{Total resistance force and power as time function}
\end{figure}

\textbf{}
% ----------------------------------- Energetical point of view
\section{Energy}
\textbf{\underline{Energy in driving cycle related to the wheel}}
\begin{equation}
E_{wheel} = \int_{0}^{t_{end}} P_{wheel}*dt
\end{equation}

\begin{figure}[H]
  \centerline{\includegraphics[width=\textwidth]{Energy.png}} 
  \caption{Energy related to the wheels as time function}
\end{figure}


\section{Wheel / Motor}
\begin{equation}
\omega_{wheel} = \frac{V_{vehicle}}{r_d}
\end{equation}

\begin{equation}
M_{wheel} = F_{drive}*r_d
\end{equation}

\begin{equation}
\omega_{motor} = k*\omega_{wheel}
\end{equation}


\[\begin{equation}
    M_{motor}= 
\begin{dcases}
    \frac{M_{wheel}}{k}\eta^{-1},&  M_{wheel}*\omega_{wheel} > 0\\
    \frac{M_{wheel}}{k}\eta,&         M_{wheel}*\omega_{wheel} < 0
\end{dcases}
\end{equation}\]


\begin{equation}
i = \frac{M + (c_2*\omega^2 + c_1*\omega + c_0)}{c*\phi}
\end{equation}

\begin{equation}
U_{motor} = R*i + c*\phi*\omega
\end{equation}

\begin{figure}[H]
  \centerline{\includegraphics[width=\textwidth]{motor_results.png}} 
  \caption{Motor results}
\end{figure}

% ---------------------------------------------------------------
\textbf{\underline{Model of inverter}}

\[\begin{equation}
    P_{battery}= 
\begin{dcases}
    U_{motor}*i*\eta^{-1},&  i > 0\\
    U_{motor}*i*\eta,&        i < 0
\end{dcases}
\end{equation}\]


\begin{equation}
duty = \frac{U_{motor}}{U_{battery}}
\end{equation}

\section{Model of battery}
\begin{figure}[H]
  \centerline{\includegraphics{battery_model.PNG}} 
  \caption{Battery model}
\end{figure}

\begin{equation}
E_{bat} = U_{bat} - R_{wew}*I_{bat}
\end{equation}




\end{document}