\documentclass[a4paper,class=article,border=10pt,tikz]{standalone}

%mypackages
\usepackage{pythontex}
\usepackage{pgfplots}
\usepackage{amsmath}
\usepackage{titlesec}
\usepackage{tikz}
\usetikzlibrary{shapes.geometric}
\usetikzlibrary{positioning}
\usetikzlibrary{snakes,calc,positioning,patterns,angles,quotes,decorations.pathmorphing,decorations.markings}
% \titleformat{<command>}[<shape>]{<format>}{<label>}{<sep>}{<before-code>}[<after-code>]
%\titleformat{\section}{\normalfont\Large\bfseries}{\thesection.}{10pt}{}
% \titlespacing{<command>}{<left>}{<before-sep>}{<after-sep>}
%\titlespacing{\section}{0pt}{14pt}{7pt}

%\titleformat{\subsection}{\normalfont\itshape}{\thesubsection.}{10pt}{}
%\titlespacing{\subsection}{0pt}{12pt}{6pt}
% set font encoding for PDFLaTeX, XeLaTeX, or LuaTeX
\usepackage{ifxetex,ifluatex}
\if\ifxetex T\else\ifluatex T\else F\fi\fi T%
  \usepackage{fontspec}
\else
  \usepackage[T1]{fontenc}
  \usepackage[utf8]{inputenc}
  \usepackage{lmodern}
\fi

\usepackage{hyperref}


\title{Title of Document}
\author{Name of Author}

% Enable SageTeX to run SageMath code right inside this LaTeX file.
% http://doc.sagemath.org/html/en/tutorial/sagetex.html
% \usepackage{sagetex}

% Enable PythonTeX to run Python – https://ctan.org/pkg/pythontex
% \usepackage{pythontex}

\begin{document}

\begin{tikzpicture}[scale=1.5, every node/.style={scale=1.5},point/.style = {draw, circle, fill=black, minimum size=3pt, inner sep=0.5pt},]
 [
   scale=1,
   point/.style = {draw, circle, fill=black, minimum size=3pt, inner sep=0.5pt},
 ]

% x, y axis
  \draw[->] (-4,0) -- (5,0) node[right] {$x$};
  \draw[->] (0,-4) -- (0,5) node[above] {$y$};
 % Origin
  \coordinate (O) at (0,0);

  \def\rad{2.5cm}
  \node (O) at (0,0) {};
  \node [above left] at (0,0) {$C_0$};
  \draw (O) circle (\rad);
  \node (P)  at +(25:\rad)  [point]{};
  \node (P_label) at ($(P.north)+(9pt,3pt)$) {$A$};

% tangent line
  \draw [thick,black] (P) -- ($(P)!2.25!90:(O)$) coordinate (bob) node[near end, right]{$m_{1}$, $l_{1}$};
  \draw [thick,gray,dashed] (P) -- ++(-90:4) coordinate (line);
  \fill (bob) circle (0.2);

  \pic [draw, ->, "$\varphi$", angle eccentricity=1.6] {angle = line--P--bob};
\draw[<->,line width=1.5pt] (-0.5,0)--(0.5,0) node[midway,below,xshift=0.6cm]{$x_{{C0}}(t)$};
\fill[black] (O) circle (0.05);
\end{tikzpicture}
\end{document}
% \draw[<->,line width=1.5pt] (-0.5,0)--(0.5,0) node[midway,above]{$x_{e}(t)$};